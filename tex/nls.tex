\documentclass{article}
\usepackage{amsfonts}
\begin{document}

\newcommand{\op}[1]{\mathsf{#1}}
\newcommand{\cnt}{c}
\newcommand{\scst}{\sigma}
\newcommand{\dyst}{\delta}
\newcommand{\pest}{\pi}
\newcommand{\sym}{\iota}
\newcommand{\prop}[1]{\mathsf{#1}}
\newcommand{\fun}[1]{\mathsf{#1}}
\newcommand{\type}[1]{\mathcal{#1}}
\newcommand{\val}[1]{\mathsf{#1}}
\newcommand{\eval}{\mathcal{E}}
\newcommand{\syn}[1]{\mathsf{#1}}
\newcommand{\bif}{\quad \mathbf{if} \quad}

\section{Formal semantics}

The brace syntax, ie. $\{ a \in \mathbb{N}, b \in \mathbb{Z} \}$, denotes a
record with two named fields, $a$ and $b$. (Probably need less confusing syntax
for this).

\subsection{Definitions}

\[ \iota \in \type{I} \]

An \emph{identity}, and opaque symbol that identifies some object or program
component. Identities can be compare for equality and an identity is equal only
to itself.

\begin{eqnarray*}
v \in \type{V}
& \in & \val{integer}(\mathbb{Z}) \\
& | & \val{string}(\Sigma^*) \\
& | & \val{lambda}(\type{S}, \type{I}^*, \type{A}) \\
& | & \val{object}(\type{I}) \\
& | & \ldots
\end{eqnarray*}

Values.

\begin{eqnarray*}
a \in \type{A}
& \in & \syn{with\_escape}(\type{I}, \type{A}) \\
& | & \syn{ensure}(\type{A}, \type{A}) \\
& | & \syn{escape}(\type{I}, \type{A}) \\
& | & \syn{variable}(\type{I}) \\
& | & \ldots
\end{eqnarray*}

Abstract syntax.

\[ \cnt \in \type{C} = \type{V} \mapsto \type{P} \mapsto \type{V} \]

A continuation. Accepts the result of the immediately preceding evaluation and
the current pervasive state and produces another value, the result of continuing
the computation.

\[ \scst \in \type{S} = \{ b \in (\type{I} \times \type{V})^* \} \]

The \emph{lexical scope} state which encapsulates the locally scoped context of a
computation, the state that only depends on lexically enclosing information. The
lexical state is not passed from a caller into a method being called, for
instance.

The $b$ component is the bindings visible in the given scope, a mapping from
variables by identity to values.

\[ \dyst \in \type{D} = \{ e \in \type{I} \mapsto \type{C} \} \]

The \emph{dynamic scope} state encapsulates the dynamic context of a computation.
Being dynamically scoped means that it may be passed from a caller into a method
being called but never backwards, from a called method into the caller.

The $e$ component is the behavior of a non-local escape. The first argument is
the identity of the destination to escape to.

\[ \pest \in \type{P} = \{ i \subset \type{I} \} \]

The \emph{pervasive} state encapsulates unconstrained state. Computational
side-effects operate on the pervasive state. The pervasive state is passed
linearly through each step of the computation so the effects of a method call on
the pervasive state is visible to the caller, unlike the lexically and dynamically
scoped state.

The pervasive state is immutable, the effect of changing state is obtained by
producing a new pervasive state with the change applied and using that going
forward, discarding the old state. The new state is said to be \emph{downstream}
of the old state. This is how evaluation order is made explicit: given two
expressions $a_0$ and $a_1$ which are evaluated in pervasive states $\pest_0$
and $\pest_1$ respectively, $a_1$ is evaluated after $a_0$ exactly if $\pest_1$
is downstream of $\pest_0$.

The $i$ component is the finite set of identities that have already been used
during the evaluation.

\[ \eval \in \type{A} \mapsto \type{C} \mapsto \type{S} \mapsto \type{D} \mapsto \type{P} \mapsto \type{V} \]

The evaluation operator which accepts an expression, a continuation, and one of
each of the three kinds of environment an evaluation can take place within.

\subsection{Utilities}

\begin{eqnarray*}
\fun{gensym}(\pest_0) & = & (\sym, \pest_1) \\
\sym & \notin & \pest_0.s \\
\pest_1 & = & \pest_0 / \{ s = \pest_0.s \cup \{ \sym \} \}
\end{eqnarray*}

The $\fun{gensym}$ function accepts a pervasive state and yields a fresh symbol
and a new pervasive state to use from that point on. It is guaranteed that the
same symbol will never be returned from the new pervasive state or any states
further downstream.

\subsection{Non-local escapes}

Three forms deal with non-local escaping, that is, aborting a sub-computation
abruptly and returning control to an arbitrarily distant enclosing computation.
They are: $\syn{with\_escape}$ which sets up a location that can be escaped to,
$\syn{escape}$ which initiates an escape to a particular location, and
$\syn{ensure}$ which registers an expression to always be evaluated however
the body of the expression completes, normally or by escaping.

\subsubsection{With-escape}

The $\syn{with\_escape}(\sym, a) $ form creates a \emph{non-local escape} object
that, when called, attempts to return control to the $\syn{with\_escape}$ form
that created it. The escape object's call method accepts a value which will be
used as the value of the $\syn{with\_escape}$ expression. If the body completes
normally the value of the whole expression will be the value of the body expression.

Here's an example of a typical use,

\begin{verbatim}
def $result := with_escape ($break) {
  for ($k, $v) in $elements do {
    if $k == $key
      then $break($v);
  }
  null;
}
\end{verbatim}

This code iterates through key/value pairs from the collection \texttt{\$elements}
and if a pair if found whose key matches \texttt{\$key}, the key we're looking for,
we immediately return the value as the value of the expression. If no key is
found we don't need to escape and the computation completes normally with value
\texttt{null}.

A $\syn{with\_escape}$ can only be successfully escaped once and leaving the
expression normally is considered as a successful escape.

\begin{eqnarray*}
\eval(\syn{with\_escape}(\sym, a), \cnt, \scst_0, \dyst_0, \pest_0) & \to & 
\eval(e, \cnt, \scst_1, \dyst_1, \pest_1)
\end{eqnarray*}

where,

\begin{eqnarray*}
(\sym_e, \pest_1) & = & \fun{genuid}(\pest_0) \\
\dyst_1 & = & \dyst_0 / \{ e = e_e \} \\
e_e(\sym_t) & = & c \bif \sym_t = \sym_e \\
e_e(\sym_t) & = & \dyst_0.e(\sym_t) \bif \sym_t \neq \sym_e \\
\scst_1 & = & \scst_0 / \{ b = b_e \} \\
b_e & = & (\sym, v_e) :: \scst_0.b \\
v_e & = & \val{lambda}(\bot, \sym_l, \syn{escape}(\sym_e, \syn{variable}(\sym_l)))
\end{eqnarray*}

The symbol $\sym_e$ identifies this escape location and is used when escaping
to decide when we've reached the target. $e_e$ is the escape function to be
used in the body of the expression. If it is used to escape to this expression
it aborts by immediately calling the expression's continuation. If it is not
the target it continues propagating. $v_e$ holds the escape object which is made
available to the body of the expression through the name $\sym$ in the new
lexical scope used for the body.

\subsubsection{Ensure}

The $\syn{ensure}(a_b, a_e) $ form evaluates its body expression, $a_b$ in a
context where however the evaluation completes, normally or through a non-local
escape, the ensure expression $a_e $ is guaranteed to be executed. If the body
completes normally the value of the entire expression will be the body's value;
the ensure block will be evaluated but the value will always be discarded.

\begin{eqnarray*}
\eval(\syn{ensure}(a_b, a_e), \cnt, \scst_0, \dyst_0, \pest_0) & \to &
\eval(a_b, c_n, \scst_0, \dyst_1, \pest_0)
\end{eqnarray*}

where

\begin{eqnarray*}
\dyst_1 & = & \dyst_0 / \{ e = e_e \} \\
e_e(\sym_t, v^e_b, \pest^e_1) & = & \eval(a_e, \dyst_0.e(\sym_t), \scst_0, \dyst_0, \pest^e_1) \\
\cnt_n(v^n_b, \pest^n_1) & = & \eval(a_e, c_d, \scst_0, \dyst_0, \pest^n_1) \\
\cnt_d(v_e, \pest^n_2) & = & \cnt(v^n_b, \pest^n_2)
\end{eqnarray*}

There are two ways the body can complete, either normally or by escaping. The
$e_e$ escape function handles the abrupt case: if the body escapes $e_e$ will
be called and it will evaluate the ensure block. It passes the enclosing escape
continuation such that on completion the escape process will continue. The ensure
block is evaluated in the original dynamic scope such that if it escapes it is
not evaluated again.

The normal case is handled by $\cnt_e$ which evaluates the ensure block, then
discards the result (that's the purpose of $\cnt_d$), and finally yields the
value of the block as the result of the whole expression.

\subsubsection{Escape}

The $\syn{escape}(\sym, a)$ form is synthetic and only produced by the
$\syn{with\_escape}$ form. The symbol $\sym$ identifies where the expression
will attempt to escape to and the expression $a$ provides the value to escape
with.

The escape process takes care to call any ensure blocks between where it is
called and its destination. Since ensure blocks contain arbitrary code they
themselves can initiate non-local escapes, in which case the non-local escape
process that caused the ensure block to be evaluated will be abandoned. Because
of this an escape can be invoked meaningfully more than once as long as the
previous calls have all been interrupted in this way before they could succeed.

\begin{eqnarray*}
\eval(\syn{escape}(\sym, a), \cnt, \scst_0, \dyst_0, \pest_0) & \to &
\eval(a, \dyst_0.e(\sym), \scst_0, \dyst_0, \pest_0)
\end{eqnarray*}

Escaping abandons its normal continuation and replaces it in the evaluation of
the value with the current non-local escape continuation, binding the destination
to the escape's target.


\end{document}
