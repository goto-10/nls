\documentclass{article} 
\begin{document}

\newcommand{\eval}{\mathsf{E}}
\newcommand{\op}[1]{\mathsf{#1}}
\newcommand{\cnt}{c}
\newcommand{\scst}{\sigma}
\newcommand{\dyst}{\delta}
\newcommand{\pest}{\pi}
\newcommand{\sym}{\nu}
\newcommand{\prop}[1]{\mathsf{#1}}
\newcommand{\fun}[1]{\mathsf{#1}}

\[ \sym \in N \]

An opaque symbol that identifies some object or program component. Symbols can
be compare for equality and a symbol is equal only to itself.

\[ \cnt \in C = V \times \Pi \mapsto V \]

A one-step continuation. Accepts a value to operate on and the current pervasive
state and produces another value, the result of continuing the computation.

\[ \scst \in \Sigma = \{ \} \]

A local scope state encapsulates the lexically scoped context of a computation.
Being locally scoped means depending only on lexically enclosing information;
the locally scoped state is not passed from a caller into a method being called,
for instance.

\[ \dyst \in \Delta = \{ E \in N \mapsto C \} \]

The dynamic scope state encapsulates the dynamic context of a computation. Being
dynamically scoped means that it may be passed from a caller into a method being
called but never backwards, from a called method into the caller.

The $E$ component is the behavior of a non-local escape. The first argument is
a symbol that identifies the destination to escape to.

\[ \pest \in \Pi = \{ S \subset N \} \]

The pervasive state encapsulates unconstrained state. Computational side-effects
generally operate on the pervasive state. The pervasive state is passed linearly
through each step of the computation so the effects of a method call on the
pervasive state is visible to the caller, unlike the local and dynamic state.

The $S$ component is the finite set of symbols that have previously been used
during the evaluation.

\begin{eqnarray}
\fun{gensym}(\pest_0) & = & (\sym, \pest_1) \\
\sym & \notin & \pest_0.S \\
\pest_1 & = & \pest_0 / \{ S: \pest_0.S \oplus \{ \sym \} \}
\end{eqnarray}

The gensym function accepts a pervasive state and yields a fresh symbol and a
new pervasive state to use from that point on. It is guaranteed that the same
symbol will never be returned from the resulting pervasive state or any
other downstream pervasive states.

\begin{eqnarray}
\eval(\op{with\_escape}(\sym, e), \cnt_0, \scst_0, \dyst_0, \pest_0) & \to & 
\eval(e, \cnt_1, \scst_1, \dyst_1, \pest_1)
\end{eqnarray}

\end{document}
